% !TeX root = ../main.tex

\begin{thebibliography}{99}
  \bibitem{linjin}秦微. 基于雷达和闪电资料的雷电临近预警对比分析[A]. 中国气象学会.第32届中国气象学会年会S20 第十三届防雷减灾论坛——雷电物理和防雷新技术[C].
  中国气象学会:中国气象学会,2015:3.
  \bibitem{leida}阎访,陈静,卞韬,廖颖慧,张翠华.一次雷暴大风的物理环境场和多普勒雷达回波特征[J].气象与环境学报,2013,29(01):33-39.
  \bibitem{fenlei1}A. Heinle,A. Macke,A. Srivastav. Automatic cloud classification of whole sky images[J]. 
  Atmospheric Measurement Techniques,2010,3(3).
  \bibitem{fenlei2}A. Kazantzidis,P. Tzoumanikas,A.F. Bais,S. Fotopoulos,G. Economou. Cloud detection and 
  classification with the use of whole-sky ground-based images[J]. Atmospheric Research,2012,113.
  \bibitem{huidugongshengjuzhen}苑丽红,付丽,杨勇,苗静.灰度共生矩阵提取纹理特征的实验结果分析[J]
  .计算机应用,2009,29(04):1018-1021.
  \bibitem{glcm}Tamura H, Mori S, Yamawaki T. Textural features corresponding to visual perception[J]. IEEE Transactions on Systems, man, and cybernetics, 1978, 8(6): 460-473.
  \bibitem{gaofenbianlv}谭凯. 高分辨率遥感卫星影像自动云检测算法研究[D].武汉大学,2017.
  \bibitem{pinlv}陈树彬,王小铭.一种基于彩色图像频率特征的颜色量化算法[J].华南师范大学学报(自然科学版),2009(03):36-38.
  \bibitem{lunkuofa}刘永禄,邵利民,杨汶鑫.基于卫星云图的轮廓法自动识别热带气旋研究[J].海洋预报,2012,29(01):13-17.
  \bibitem{lunkuofa_eng}Pan J L, Zhu M, Zhu J H, et al. An algorithm using satellite image data for the forecasting of tropical cyclone path[J]. Journal of
  Oceanography in Taiwan Strait, 2009, 22(3): 425-432.
  \bibitem{jihetezheng}李建更,贾真,阮晓钢.基于图像几何特征的行星表面参考区域提取与匹配方法[J].
  北京工业大学学报,2014,40(07):1066-1072.
  \bibitem{liangwencha}王寅钧,陈渭民,吴彬.一次雷暴天气发生发展的水汽图和红外云图特征分析[J].自然灾害学报,2012,21(02):126-134.
  \bibitem{FY2D}周晓丽,杨昌军.基于FY-2D的新疆区域强对流云识别[J].沙漠与绿洲气象,2017,11(02):82-87.
  \bibitem{duochidu}杜坤. 多尺度资料在强对流天气预报中的应用[D].南京信息工程大学,2011.
  \bibitem{duochidu_eng}Moller A R, Doswell III C A, Foster M P, et al. The operational recognition of supercell thunderstorm environments and storm structures[J]. Weather and Forecasting, 1994, 9(3): 327-347.
  \bibitem{FY4A}王淦泉,沈霞.风云四号辐射成像仪及其数据在卫星气象中的应用[J].自然杂志,2018,40(01):1-11.
  \bibitem{fengyunsihao}曹冬杰.风云四号静止卫星闪电成像仪监测原理和产品算法研究进展[J].气象科技进展,2016,6(01):94-98.
  \bibitem{lmi}曹冬杰,陆风,张晓虎,张志清.风云四号卫星闪电探测产品在强对流天气监测中的应用[J].卫星应用,2018(11):18-23.
  \bibitem{shandianyi}Christian H J, Blakeslee R J, Goodman S J. The detection of lightning from geostationary orbit[J]. Journal of Geophysical Research: Atmospheres, 1989, 94(D11): 13329-13337.
  \bibitem{zhuchengfen}李成龙,张景发.基于主成分分析的遥感震害变化检测方法与应用[J].地震,2013,33(02):103-108.
\end{thebibliography}
