% !TeX root = ../main.tex

\chapter{卫星影像云图特征}
近年来,卫星云图处理技术得到了很大的发展,研究主要集中在对云分类和判断云型方面\cite{fenlei1}\cite{fenlei2}。
这些处理方法涵盖模式识别、图像处理等多学科范畴,已成为卫星云图处理技术研究中很重要的一部分。
通常分为云图预处理,图像分割,目标特征提取等几个步骤。

由于卫星在观测及信号传输的过程中会受到大气等传输媒介的干扰影响,
卫星云图数据都具有一些“噪声”,云图预处理主要用于清除产生影 响的“噪声”。常用的预处理手段有:
平滑滤波,伪彩色处理,卫星云图直方图增强等等。

图像分割是将一幅气象卫星云图分成若干与实际目标物体对应的子图,
所获得的子图可以用于下一步图像处理中,也可以进行一些特征的解释和识别。
常用的云图图像分割方式有:多谱阈值法,数学形态法和神经网络法等方法。

经过预处理和分割后的像素集需要以计算机可以理解的方式来表示和描述,常用的描述方式有:
选择其外部特征(边界)来表示区域,或根据其内部特征(该区域的的像素值)来表示。
常见的目标特征有纹理,频率,几何,主分量等特征。本文又针对雷暴云的物理特性,
概括总结了与强对流天气相关的量温差、有效对流位能和抬升指数等物理量和指数。

\section{云图的纹理特征}
卫星云图属于自然纹理,具有随机性和多样性。卫星云图中灰度信息的变化情况可由纹理信息来描述,
可以反映云的状态变化,也可反映其自身性质。
由于云的云内气流、大气环流、水汽含量等差异,致使云的密度、云顶高度及形态各不相同,
在图像中反映的纹理也具有多样性。在视觉上,云的纹理可以分为平整与起伏、粗糙与平滑、规则与杂乱等多种情况,
常用灰度共生矩阵来描述纹理特征。灰度共生矩阵来描述纹理的方法是20世纪70年代初由 R.Haralick 等人
提出的具有广泛性的纹理分析方法。

\subsection{灰度共生矩阵的定义}
设图像水平和垂直方向上各有$N_c \times N_r$个像元,将每个像元出现的灰度量化为$N_g$层,
设$L_x = {1,2,...,N_c}$为水平空间域,$L_y = {1,2,...,N_r}$
为垂直空间域,$G = {1,2,...,N_g}$ G = { 1, 2, …, N g}为量化灰度层集。集$L_x \times L_y$
为行列编序的图像像元集,则图像函数$f$可表示为一个函数:指定每一个像元具有$N_g$个灰度层中的一个值$G$,
即$f:L_x \times L_y \rightarrow G$。灰度共生矩阵定义为在图像域$L_x \times L_y$范围内,
两个相距为$d$,方向为$θ$的像元在图像中出现的概率,即:
\begin{equation}
\begin{aligned} P(i, j | d, \theta)=& \#\{[(k, l),(m, n)] \in(L x \times L y) \times\\ &(L x \times L y) | k-m=d, l-n=\\ &-d, f(k, l)=i, f(m, n)=j \} \end{aligned}
\end{equation}

灰度共生矩阵提供了图像灰度间隔、变化方向和幅度的信息,可根据共生矩阵来计算一些对应的特征值,
图像的纹理信息可用特征值来表示\cite{huidugongshengjuzhen}\cite{glcm}。

\subsection{灰度共生矩阵量化值}
灰度共生矩阵理论研究当中往往通过角二阶矩ASM、对比度CON、逆差距IDM、熵ENT等物理量对其进行量化。

\subsubsection{角二阶矩}
\begin{equation}
    ASM=\sum_{i=0}^{N g-1} \sum_{j=0}^{N g-1}(P(i, j))^{2}
\end{equation}
如果影像中噪声越多,则ASM值就越小,代表影像纹理越丰富。如果影像中地物灰度分布均匀(例如,云层、水体),
则ASM值越大,纹理信息越弱。

\subsubsection{对比度}
\begin{equation}
    CON=\sum_{n=0}^{N g-1} n^{2}(\sum_{i=0}^{N g-1} \sum_{j=0}^{N g-1} P(i, j))
\end{equation}
CON对比度描述的是影像中像素与其周围像素的反差对比。对比度越小表明该区域的像素灰度越均匀纹理越弱,
反之纹理越丰富。

\subsubsection{逆差距}
\begin{equation}
    IDM=\sum_{i=0}^{N g-1} \sum_{j=0}^{N g-1} \frac{P(i, j)}{1+(i-j)^{2}}
\end{equation}
IDM逆差距是反映图像中某一区域内像元值变化程度的量。IDM值越大,表明图像中该区域的纹理越弱,
反之,纹理越丰富。

\subsubsection{熵}
\begin{equation}
    ENG=-\sum_{i=0}^{N g-1} \sum_{j=0}^{N g-1} P(i, j) \ln P(i, j)
\end{equation}
熵原本用来描述分子不规则运动剧烈程度的物理量,后来用来度量影像中所包含的纹理信息。
熵值越大,表明影像中所包含的纹理信息越丰富,反之,纹理信息越弱。






\section{云图的频率特征}
影像频率是描述影像中光谱信息变化程度的的物理量。例如,水面或者无云的天空,该类型的目标面积广泛,
并且灰度分布均匀,纹理较弱,因此对应的频率较低。而房屋或者人工建筑区,光谱信息变化剧烈,
纹理信息丰富,因此对应的频率较高。对于影像中的云层而言,云层边缘灰度突变,因此分布在高频部分,
而云层内部灰度均匀,因此分布在低频部分。频率域为评价影像提供了一个全新的角度,
而傅里叶变换以及小波变换是将影像由空间域转换到频率域的两个重要算法\cite{gaofenbianlv}\cite{pinlv}。

\subsection{傅里叶变换}
图像频率本质上也是灰度分布情况决定的。通过傅立叶变换可以将图像从空间域转到频率域。
而逆傅立叶变换可以将图像从频率域转回到空间域。换言之,该变换可从图像光谱分布规律中提取信号频率分布规律,
而其逆变换则是从图像频率分布规律中提取光谱分布规律。在实际图像处理当中,往往将图像当作二维矩阵,
采用二维傅立叶变换进行图像频率域信息提取。

\subsection{小波变换}
小波变换是局部的空间域与频率域的转换算法,鲁棒性较强。与傅立叶变换算法相比,
该算法通过平移、伸缩变换对图像进行了多尺度分析,处理效果更好。
与傅立叶变换类似,为了在频率域中进行云检测研究,研究中在小波变换后的频谱图中,
滤除高频亮点,保留低频暗点,然后再将滤波后的频谱图进行逆变换,以达到剔除高分影像中非云噪声的目的,
并且该过程是需要迭代进行,通过多次小波分解,直至满足需求。




\section{云图的几何特征}
常用的几种提取几何特征的方法有:SIFT特征算子,SURF特征算子,Hough变换,LSD线特征等方法。 
卫星云图上的各类云系都具有不同的边界形状,云的边界是判断天气系统的重要依据,如成熟的台风呈圆形,
冷锋云带呈气旋形弯曲。形状的描述主要分为以下几种方法:基于面积、伸长度、主轴方向等传统的形状特征;
基于形状变换的方法;基于形状相互关系的方法\cite{lunkuofa}\cite{lunkuofa_eng}。 

对于封闭的几何图形,可用点状特征、线状特征和面状特征来描述。
这里着重讨论线状特征和面状特征\cite{jihetezheng}。

\subsection{线状特征}
\subsubsection{周长$C$}
一般来讲,边界轮廓的像素点数即为周长,轮廓周长即为白色像素点个数。

\subsubsection{最小外接矩形周长$L_m$}
一般利用图形边界的最大坐标值和最小坐标值可以求出目标图形的外接矩形,但是,
拍摄的图像如果有一定的旋转角度,外接矩形则会发生很大的变化。
因此,目标图形的最小面积外接矩形用来描述几何特征是很有必要的。
则最小外接矩形的周长定义为 $L_m$。

\subsection{面状特征}
\subsubsection{面积$S$}
二值图像中对于某个区域$R$来说,区域中白色像素的个数可以来表示面积。
面积可利用顺序扫描的方式计算出来:
\begin{equation}
    S=\sum_{x} \sum_{y} g(x, y),(x, y) \in R
\end{equation}

\subsubsection{最小外接矩形面积 $A_M$}
\begin{equation}
    A_{\mathrm{M}}=L_{\mathrm{M}} \times W_{\mathrm{M}}
\end{equation}
式中, $L_M$和 $W_M$分别为最小外接矩形的长和宽。 

\subsubsection{形状参数$F$}
根据边界的周长$C$和区域的面积$S$计算得到形状参数为
\begin{equation}
    F=\frac{C^{2}}{4 \pi S}
\end{equation}



\section{雷暴相关物理量和指数}

\subsection{亮温差}
某一波长发射体的光谱辐亮度和同一波长下的黑体光谱辐亮度相等时,
把发射体的辐亮度温度称为黑体温度。如果波长在可见光谱范围内,用具有人眼光谱光视效率响应的传感器
来判断其亮度相等时,则称为亮度温度,简称为亮温。

云顶亮温信息可由卫星红外通道探测到,可以反映降水云团的辐射特性,揭示云的物理性质。 
对于红外云图,暖云的辐射强度要高于冷云,云团内部的上升气流加速雨滴凝结,
同时抬升云顶高度,温度下降,云中的降水情况可由云顶温度的变化反推出来。 
红外云图上的对流性降水云系,表现为亮度温度低,上升运动越强则云顶高度越高,又由于水汽在大气中含量丰富,
提高了降雨的可能性\cite{liangwencha}。

在水汽通道中,由于下沉和上升运动都可以造成明显的湿度变化,反映了大气中高云的信息和高层的干湿状况。
大气中细微水滴借助对流层中上层的垂直运动凝聚,辐射值也随之增高,
与红外通道的差值就会变小,因此利用水汽通道和红外通道的亮温差进行分析,
判断对流云团的云顶高度和云层厚度有很大的指示意义\cite{FY2D}。

\iffalse
在红外$10.7\sim12.0μm$波段之间,由于红外一通道的下降趋势相比红外二通道要明显,
故通道之间的差值可以用来描述强对流发生过程中的发展状态。对于积状云边缘的卷云结构,虽然云顶亮温很低,
但云的光学厚度较小,不足以遮挡云底的向上辐射,造成两个通道上的亮度温度值相比较$CH1> CH2$,
半透明云区或部分有云区存在亮温差值较大,而在对流发展较强的区域,云层越厚,云顶亮温越低,
两通道的亮温差偏小,说明红外一通道与二通道亮温差提供了一定的降水信息。
\fi

\subsection{有效对流位能}
有效对流位能(Convective available potential energy)是大气科学当中使用的量值,
为评估对流是否容易发展、垂直大气是否稳定的指标之一。
近地面的空气块受垂直风切扰动或地形等其他因素而沿着绝热线上升时,在一定高度以上其温度若比周围环境温度高,
意味着气块密度较周围环境空气小,则周围环境将给予气块向上的浮力。
周围环境对空气块的作用力与空气块位移相乘,而得到周围环境对气块所做的功,
这部分的能量在理想状态下将会储存在空气块中,使其具有向上发展的动能。
一般对流可用位能的计算范围,是以自由对流高度以上到平衡高度为止,周围环境所能提供的浮力对高度积分而得。


\subsection{抬升指数}
抬升指数(lifting index)表示自由对流高度以上不稳定能量面积(从热力图表中得出)大小的指数。
差值为正时,数值愈大,表示正的不稳定能量面积意大,出现对流的可能性愈大。
大气层结稳定时差值为负。有时还使用最有利抬升指数(benefit liftingindex)。
它的求法是把700hPa以下的大气,按50hPa间隔分为许多层,并将各层中间高度处上的各点,
分别沿于绝热线抬升到各自的凝结高度,然后又分别沿湿绝热线抬升到500hPa,于是可分别得到各点的抬升指数,
其中正值最大者即为最有利抬升指数\cite{duochidu}\cite{duochidu_eng}。


\section{小结}
图像特征种类繁多,包含边界特征,几何特征,频率特征和纹理特征等,
根据雷暴云的物理特性,本章主要介绍了几种针对雷暴云的图像特征,其中纹理特征是本文主要使用的方法,
频率特征可以提取出云图的高频和低频信息,几何特征可以界定雷暴云的范围,主分量特征用于降维处理,
并在此基础上介绍了亮温差,有效对流位能,抬升指数等几种用于判断强对流天气的指数和物理量。