% !TeX root = . . /main. tex

\begin{abstract}
  常规的雷暴探测,主要借助自动气象站、闪电定位仪、雷达、探空等技术手段,但是,
  由于雷暴发生的时空尺度较小,常规气象探测手段的短期临近预报能力较弱,
  对雷暴、强风、强降雨等灾害天气的虚报漏报率较高。
  通过对高精度的风云四号卫星数据进行处理,充分挖掘云图中与雷暴区域相关性强的云区信息,
  从图像角度提取雷暴区域云图的相关特征,利用数字图像处理理论,本文提出一种根据纹理特征阈值判断雷暴云的方法,
  从而提高对雷暴天气的监测和预警能力。经测试库测试,本文提出的方法,预警率可达到90$\%$以上。

  \keywords{风云四号;特征提取;雷暴预警}
\end{abstract}

\begin{enabstract}
  \iffalse
  Conventional thunderstorm detection, mainly by means of automatic weather stations, 
  lightning locators, radar, sounding and other technical means.  However, due to the small spatial and temporal scale of thunderstorms, 
  the short-term nowcasting capability of conventional meteorological means is still 
  very low. The false report and false negative rate of disaster weather 
  such as thunderstorms, strong winds and heavy rainfall are still high. This article uses digital image processing theory 
  by processing the high-precision FY4A satellite data. The cloud area feature information with strong correlation 
  with the thunderstorm area in the cloud map is fully tapped. The related features of thunderstorm region cloud map are extracted from the image, 
  and a method for judging thunderstorm cloud based on texture feature threshold is proposed. 
  Improving monitoring and early warning capabilities for thunderstorms. 
  After the test library test, the method proposed in this paper can reach more than 90$\%$. 
\fi

  Conventional thunderstorm detection, mainly by automatic weather stations, lightning locators, radar, sounding and other technical means. However, due to the small spatial and temporal scale of thunderstorms, the short-term nowcasting capability of conventional meteorological means is still 
  weak. The rate of false positives and false negatives of disaster weather 
  such as thunderstorms, strong winds and heavy rainfall are still high. With 
  processing the high-precision FY4A satellite data, digging thunderstorm-correlative cloud area information and extracting 
  image features of the cloud map of thunderstorm area, this paper, combined with digital image processing theory, 
  comes up with a method for judging thunderstorm cloud based on texture feature threshold to improve monitoring and early 
  warning capabilities for thunderstorms. Testing in the test library, the method proposed in this paper can reach an accuracy of more than 90$\%$. 
  \enkeywords{FY4A; feature extraction; thunderstorm warning}
\end{enabstract}
