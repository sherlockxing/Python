% !TeX root = ../main.tex

\chapter{总结与展望}
本文立足风云四号卫星影像数据,发挥气象卫星的优势,弥补地面设备雷电监测的缺陷,
讨论了卫星影像的图像特征以及针对风云四号云图雷暴提取的相关问题,
在建立图像样本库,计算图像特征数据的基础上,提出了雷暴云图的图像特征阈值范围。

基于地面设备的雷电监测,由于雷电的时空尺度较小等因素,目前难以实现实时对闪电的跟踪观测,
气象卫星则可以克服此缺陷,因此本文尝试从图像角度来提取雷暴云区的相关特征。
常用的图像特征主要有纹理特征,几何特征和频率特征等,而亮温差,有效对流位能和抬升指数
等指数和物理量也可以用来反映雷暴云区的特征。在图像特征基础上,本文又针对风云四号卫星,
从传感器波段、雷暴尺度强度和特征融合等问题进行了讨论分析。之后建立了样本库,
验证了理论分析结论的正确性,并通过数据分析提出了一套联合的图像特征阈值范围。
经测试库测试,预警率达到92.5$\%$,漏判率为24.34$\%$。

该方法的预警率达到理想精度,但漏判率偏高,表明研究中仍有不足之处,需要更加深入的研究。
首先是频率特征和几何特征,由于在实验过程中发现这两种特征在区分雷暴云和非雷暴云方面
效果不好,因此没有将此写入论文,也没有进行深入的研究,在以后的研究中将会从这两方面
特征入手,提高算法效率;其次是特征数据分析方面,没有进行数据的清洗,
选取的数据可视化方法也较为单一,因此分析得出的结论会有一定误差,下一步工作中
将考虑加入卷积神经网络的方法,提高算法的准确度。