% !TeX root = ../main.tex

\chapter{绪论}

\section{研究目的及意义}
雷暴是伴有闪电和雷击的局部对流性天气,有时还伴有阵雨、暴雨、大风和冰雹,属于强对流天气系统。
雷暴天气会对建筑物、人体、电子设备等产生极大危害,干扰无线电通讯,影响飞机等航空设备的飞行安全,
因此对雷暴区域进行研究,可以增强对雷暴的监测与预警技术,有效避免雷暴天气造成的危害,具有重大的现实意义。

传统的雷暴探测,主要借助自动气象站、闪电定位仪、雷达、探空等技术手段,
但是,由于雷暴的时空尺度较小,目前的常规气象探测网难以监测跟踪,缺乏对于雷暴生命史的完整观测,
缺乏对于雷暴精细结构的资料,尤其是缺乏影响雷暴的环境要素和促发机制的精细监测,
使得当前雷暴的短期临近预报能力还很低,雷暴、强风、强降雨等灾害天气的虚报率和漏报率依旧很高\cite{linjin}。

卫星遥感资料则可以有效弥补传统雷暴探测手段的一些缺陷。利用静止气象卫星的闪电成像仪进行闪电观测,
与传统闪电观测手段相比,地球静止轨道卫星闪电探测仪具有定位精度高、时间分辨率高、探测效率高以及
可以对闪电进行连续追踪等优点。与此同时,可以利用雷暴云的物理特性进行分析处理,
比如雷暴成熟时云顶高度较高,在红外波段的卫星云图上可以通过亮温低值区来表示,
则可以利用此信息在卫星云图上识别雷暴云团。

风云四号作为我国新一代静止气象卫星,搭载多种观测仪器,包括闪电成像仪、多通道扫描成像辐射计
和干涉式大气垂直探测仪等仪器,在世界上首次实现在静止轨道成像观测,
天基闪电观测和大气垂直探测综合观测。其中闪电成像仪
为我国首次研制,采用光学成像和ccd面阵技术,对观测区域内包括云闪、云地闪、云间闪在内的总闪电进行凝视观测,
实现对雷暴系统的实时、连续跟踪和监测,为强对流天气监测、铁路、电力、民航等行业安全保障等提供服务。

本论文紧密联系专业知识,
通过对高精度的风云四号卫星数据进行处理,充分挖掘云图中与雷暴区域相关性强的云区特征信息,
利用数字图像处理理论,尝试从图像角度提取雷暴区域云图的相关特征,提高对雷暴天气的监测和预警能力。

\newpage

\section{雷电监测方法的研究现状}
雷电是一种特殊的天气现象,雷击过程产生的大电流、高电压和强电磁辐射,经常造成严重的灾害和经济损失,
随着微电子器件的普遍使用,雷击引起的灾害越来越严重,造成的损失也越来越大,
社会对雷电的监测和预警提出了更高的要求。

对于雷电,Benjamin Franklin(本杰明$\cdot$富兰克林)于18世纪中叶首次进行了系统的研究,但直到19世纪,
用相机拍摄雷电图像以及分光光谱技术成熟之后,对雷电的研究工作才有很大的进展。
经过一段时间的平稳发展,到了20世纪60年代,对于雷电的研究又开始活跃起来,
这主要是因为雷电是威胁人类生命财产安全最严重的自然灾害之一,随着经济和社会的发展,
各国政府对雷电防护越来越重视,航空,航天,电力,信息,石油,军工,厂矿,森林等行业和体育场馆等大型建设工程,
都提出了对雷电防护的要求,同时近年固体器件和焦平面技术得到了极大的发展,给雷电研究带来了新的发展机会,
气象应用希望通过对雷电的研究得出雷电与降雨的关系,尽可能的预报雷电出现的时间。我国的气象工作者,
在20世纪80年代初,利用超外差收音机的原理进行了雷电探测,在雷电等对流天气系统降水关系的分析中,
我国科研工作者做了一定工作,取得了一些成果。目前国际上较先进的方法是利用高空飞机,航天飞船,
火箭等先进工具进行探测。
下文将从地面设备和气象卫星等两个方面,对雷电监测方法进行简短的概括说明。


\subsection{地面设备的雷电监测}
雷电的记录由来已久,主要是通过目测获得雷暴起止时间等信息。但近20年来,
由于对雷电现象研究的深入,在雷电探测对地闪的定位和监测方面实现了突破,
目前实时地闪探测技术较为成熟,可以给出地闪发生的强度、回击次数、位置以及极性等信息。
随着雷电监测技术的成熟,雷电探测到的信息不仅为雷电发展过程的研究提供了数据资料,
在其他方面也得到了广泛应用。

目前对雷暴的研究主要集中在雷暴的预警、雷暴的风场结构分析、
利用多普勒雷达探测雷暴演变、以及雷暴的电场结构的观测等方面。
地面气象台站有对雷暴观测和记录,许多地区依据多年的观测资料,
开展了对雷暴日的分析和天气研究,对雷暴出现的时间变化及其天气形势等有了更深入的认识。
从探空资料能够计算出雷暴发生需要的垂直风切变和不稳定层结,在雷暴潜势预报中,
利用探空资料得到多种指数或物理量,如有效对流位能(CAPE)、深对流指数(DCI)、对流抑制能量(CIN)、
抬升指数(LI)和强天气威胁指数。多普勒天气雷达经常用来观测雷暴结构与演变,
采用其垂直累积液水含量(VIL)、风垂直分布产品(VWP)及回波强度和高度联合分析,
可帮助区分雹暴、雷暴降水和一般性降水。自本世纪初,我国许多气象台站更新了新一代多普勒天气雷达,
利用多普勒雷达开展对雷暴的研究,主要关注最大反射率、雷暴单体低层强度回波特征、
超级单体风暴中气旋最大旋转速度、雷暴单体维持时间、雷暴单体低层强度回波特征以及雷达回波发展演变过程等\cite{leida}。


\subsection{气象卫星的雷电监测}
气象科学是建立在观测数据基础上的,世界各国都建立了大量的地面气象观测站。
但是,在海洋、高山、沙漠、极地等处,观测站点总是很稀少,一些国家观测仪器的精度也难保证,
想获取全球、三维、高时空分辨率、高精度的多种气象要素的观测数据,是很难办到的。

就对地观测而言,卫星是一个非常理想的观测位置,1957年人造地球卫星发射上天后,
最初的应用就是气象观测,1960年4月1日,美国成功发射了世界上第一颗试验气象卫星泰罗斯(TIROS), 
星载相机拍摄的第一幅地球云图,清晰地显示了大西洋上空的飓风,开创了从空间探测地球的新纪元。

对地做气象观测的卫星有科学实验卫星和业务气象卫星两类,科学实验星的运行轨道多种多样,
业务气象卫星的运行轨道现大致分为两类,一类是极轨卫星,位于太阳同步轨道,轨道高度800$\sim$1000 km,
绕地球一周约为100分钟,另一类是静止卫星,
位于地球同步轨道,卫星位于赤道上空35800km高度,向东运行,周期约为24小时,相对地球任一地方保持相对静止。
极轨卫星的作用主要是获取全球、多品种、高精度、较高空间分辨率的资料。静止卫星的作用主要是获取中低纬度、
大范围、高频次的资料。二者相互补充,缺一不可。

气象卫星用遥感探测器获取对地观测数据,具有紫外、可见光、红外、微波多种波段,经过信息加工处理后,
可以得到定量的多种气象要素和地球物理参数,也可做出各种直观,漂亮的图像。例如,描写云特征的可见光、
红外云图,云的微物理及降水特性,由跟踪云和水汽运动而得到的云迹风和水汽风,大气中水汽含量、大气温度、
臭氧含量及其垂直分布,气溶胶光学厚度、大气温室气体浓度,地气系统辐射收支能量,陆地表面的温度,
海面水温海冰分布,陆面积雪、植被指数、土壤湿度、沙尘监测等。这些信息已突破传统的气压、
温度、湿度、风向、风力等气象观测的范畴。

随着空间技术和遥感技术的飞速发展,气象卫星获取多种气象要素和地球物理参数的能力,
以及其全球性、高时空分辨率等优势更加展现,这也使气象卫星遥感成为天气分析和预报、数值预报、
气候预测和全球变化研究、生态环境和灾害监测的重要手段。气象卫星从诞生之日起,
就受到了地球科学工作者和公众的高度重视,其发展呈现出勃勃生机。

\iffalse
\subsection{卫星云图处理技术}
近年来,卫星云图处理技术发展得很快,研究方向主要集中在判断云型和对云分类方面。
这些处理方法涉及图像处理、模式识别等多学科范畴,已成为卫星图像处理技术研究中发展最快的部分之一。
通常由以下几个步骤来处理: 

\subsubsection{云图预处理}
气象卫星云图资料的预处理。由于卫星在观测和信号传输的过程中会受到大气等传 输介质的干扰影响,
卫星云图数据都具有一些“噪声”,云图预处理主要用于清除产生影 响的“噪声”。常用的预处理手段有:
平滑滤波,伪彩色处理,卫星云图直方图增强等等。

\subsubsection{云图图像分割}
云图资料的图像分割。图像分割是将一幅气象卫星云图分成若干与实际目标物体对应的子图,
所获得的子图可以用于下一步图像处理中,也可以进行一些特征的解释和识别。
常用的云图图像分割方式有:

\paragraph{阈值法}
根据云图波段的不同,卫星云图的阈值分割法也有所不同。 卫星在红外窗区测量地面和云顶的红外辐射时,
当波长一定时,卫星所测辐射仅与温度有关。红外数字云顶的亮度计数值,是由以温度函数助卫星所测辐射转换而成,
云顶越高,云顶温度越低,卫星所测辐射越小,红外亮度计数值越大。由此依一定阈值可大致区分高、中、低云。
另外,在可见光窗区,卫星所测的可见光辐射就为反照率的函数,反照率与可见光数字云图的可见光亮度计数值成正比,
反照率越大,可见光亮度计数值越大,一般而言,云体也越厚密,相应有云体厚度与可见光亮度计数值的关系曲线。
一些红外光谱特性相似的云类,如前所述的薄卷云、厚卷云、多层云系和积雨云,通常具有明显不同的厚度或反照率。
这种方法简单易行,但是由于云的类型和所处的灰度级别并不是一一对应的,有可能不同的云对应同一个灰度,
或者一种云占有好几个灰度范围, 因此该方法有一定的局限性。

\paragraph{多谱阈值法}
红外资料能识别高度不同的云,可见光资料又能区分不同厚度的云,两者各有优点也各有缺点。
若将红外和可见光两通道的资料结合起来分析,必将大大改善云分类的效果。
利用红外-可见光二维光谱特征空间, 分别以红外和可见光图像作相互正交的分量构成的测量空间,
每维坐标代表相应波段图像的亮度值。两波段图像上空间坐标一致的象素,以它的红外和可见光亮度计数值作分量,
即可构成一个光谱特征向量,并对应光谱特征空间一个点。不同的云类,因其红外光谱特性和可见光光谱特性不同,
其象素在二维光谱特征空间会形成不同位置、不同形状的集群分布,对众多考察样本进行统计分析,
并确定出区分各集群的最佳阈值后,即可用框式分类法,将集群的普遍落区用矩形框包络起来,使这包含了$256^2$
个光谱特征向量的光谱特征空间分隔成为数不多的区域, 每个区域分别表示了某一种云类或地表。
利用多个通道的云图,分别确定各个通道中各类云的灰度范围,来进行云的识别,
这种方法比简单的阈值法要精确一些,但是还是避免不了阈值法盲目截取的缺点。

\paragraph{数学形态法}
数学形态学用以描述图象的结构及特征是通过在空间域上巧妙地运用7种基本运算,
即:膨胀、腐蚀、开运算、闭运算、击中、薄化、厚化,分离出云的比较平滑的边缘形状。

\paragraph{人工神经网络法}
利用神经网络的方法对云进行分类,被普遍认为是一种较好的云分类方法。
该方法已被用于卫星云图云分类和大气参数的反演,但该方法的效果与训练资料非常有关,
必须要利用典型的资料对算法进行全面的训练。

\subsubsection{目标特征提取}
云图资料中目标的特征提取。根据实际需求,用图像处理技术将卫星云图资料中各种成分的具体特征进行提取的过程。
分割后的像素集需要以计算机可以理解的方式来表示和描述,常用的描述方式有:
(1)选择其外部特征(其边界)来表示区域,或(2)根据其内部特征(组成该区域的的像素)来表示。
常见的目标特征有纹理,几何,频率,主分量等特征。
\fi

\newpage
\section{本文研究内容及结构}
本文致力于提出一种通过图像特征判别雷暴云的方法,但受限于气象资料有限以及学界并没有对于雷暴云的准确定义,
本文将风云四号闪电仪探测到的雷暴强度大于5的闪电位置界定为雷暴云的中心。
本文需要依托大量的数据才能得出相对准确的信息,因此首先是样本库的建立,
借助闪电仪数据建立雷暴云样本库,借助云检测产品建立非雷暴云样本库,
然后通过灰度共生矩阵方法对云图纹理进行提取,通过阈值法及连通区域的计算对云图的几何特征进行提取,
并从传感器波段、雷暴尺度强度和特征融合等方面进行了分析,
提取出可以界定雷暴云的纹理特征阈值范围。在以上研究基础之上,建立测试库,
根据纹理特征阈值判别是否是雷暴云,得出该方法的预警率和虚警率。本文主要研究内容和各章节安排如下:

第一章:\textbf{绪论}。主要介绍了本课题的研究意义,概括总结了基于地面设备的雷暴监测方法并指出其不足之处,
概括了近年来发展起来的气象卫星观测方法,并创新提出了从图像特征对雷暴区域进行提取的方法。

第二章:\textbf{卫星影像云图特征描述}。图像特征种类繁多,包含边界特征,几何特征,频率特征,光谱特征,纹理特征等,
根据雷暴云的物理特性,本章主要介绍了几种针对雷暴云的图像特征,其中纹理特征是本文主要使用的方法,
频率特征可以提取出云图的高频和低频信息,几何特征可以界定雷暴云的范围,
并在此基础上介绍了亮温差,深对流指数等几种用于判断强对流天气的指数。

第三章:\textbf{风云四号卫星云图雷暴提取相关问题}。本章针对风云四号卫星,从传感器波段选择、雷暴尺度强度以及特征融合等三个方面,
分析雷暴提取中可能涉及到的相关问题。

第四章:\textbf{样本库的建立及数据分析}。本章主要介绍了本文样本库的建立流程,
包含数据的下载与分类,雷暴云样本库的建立方法,非雷暴云的提取方法,图像纹理特征的计算方法,
以及数据的分析方法,在此基础上提出了雷暴云纹理特征值的阈值范围,建立测试库验证该方法的准确率。

第五章:\textbf{基于云图特征的雷暴云提取}。本章主要对第三章理论分析的结论进行实验验证,
从传感器波段、雷暴尺度强度及特征融合等几个方面进行了实验分析,
并在此基础上提出了雷暴云的纹理特征阈值。

第六章:\textbf{总结与展望}。总结论文的主要工作,并指出其中存在的不足之处,以及对于后续改进的想法。


\iffalse
本部分将从基于地面设备(气象雷达,闪电定位仪等)的雷暴探测和基于卫星遥感的雷暴监测两方面进行简要概括,
传统的探测手段已经较为成熟,但受限于雷暴的时空尺度,当前的雷暴临近预测能力还很低,
而基于卫星遥感的雷暴监测发展时间较短,针对雷暴天气的研究并不多,研究方向主要集中在判断云型和云分类方面,
由于研究方法相似,将对此方面进行一个简要的概括。
\fi